\section{Анализ предметной области}

В этом разделе будут проведен анализ предметной области. Описаны методы обеспечения защиты информации на современных процессорах. Дано понятие и характеристика доверенной среды исполнения.

\subsection{Кольца привилегий}

В целях безопасности, компоненты любой системы разделены на уровни привилегий -- кольца защиты, за реализацию которых отвечает разработчик процессора. Во всех современных системах, реализована кольцевая система уровней привилегий. От внешнего кольца к внутреннему идёт увеличение полномочий для инструкций кода, выполняемых на процессоре в данный момент (рис. \ref{fig:rings}).

\begin{figure}[h]
	\centering
	\includegraphics[width=\textwidth]{img/rings.pdf}
	\caption{Концептуальная схема представления колец защиты в современных системах}
	\label{fig:rings}
\end{figure}

Можно создавать ещё более сложную систему -- формировать ещё больше колец защиты, для ограничения каждого из компонентов системы. Однако, чем сложнее архитектура системы и чем больше количества кода в ней, тем проще злоумышленнику найти уязвимость и эксплуатировать её \cite{complex-systems}.

Главной задачей злоумышленника является получение доступ к привилегиям, которые бы позволили получить доступ к необходимым ресурсам системы. Может показаться, что архитектурно верным является решение размещать код, который отвечает за управление пользовательским данным, и конфиденциальные данные нужно исключительно на последнем кольце защиты -- ведь получить доступ туда сложнее всего. Но у такого подхода есть свои недостатки \cite{complex-systems}. Данный подход был переосмыслен -- в настоящее используется схема, когда и код, и конфиденциальные данные хранятся на одном и том же уровне, что и пользовательские предложения, однако, доступ к ним имеет только лишь процессор. Такой метод защиты информации называется анклавом или доверенной средой исполнения.

\subsection{Доверенная среда исполнения}

Доверенная среда исполнения (ДСИ) -- специальная изолированная область, предоставляющаяся процессором, которая позволяет вынести из системы часть функциональности приложений и ОС в отдельное окружение, содержимое памяти и выполняемый код в которой будут недоступны из основной системы, независимо от уровня текущих привилегий. Так, например, в ДСИ выполняется код отвечающих за реализацию различных алгоритмов шифрования, обработки закрытых ключей, паролей, процедур аутентификации и работы с конфиденциальными данными. 

В случае, если система была скомпрометирована, информация хранящаяся в ДСИ не может быть определена, и доступ к ней будет ограничен лишь внешним программным интерфейсом. В отличии от других методов защиты защиты информации, таких как например гомоморфное шифрование, аппаратная реализация ДСИ практически не влияет на производительность системы и уменьшает время разработки программного обеспечения \cite{tee}.

С другой стороны, аппаратная реализация ДСИ имеет свои недостатки:

\begin{itemize}
	\item некоторые современные процессоры имеют лишь частичную поддержку ДСИ либо не имеют её вовсе;
	\item нет возможности программно исправить уязвимость. Найденные уязвимости в реализации ДСИ могут быть исправлены лишь в новых ревизиях процессора, т.е. без его физической замены, злоумышленник сможет эксплуатировать уязвимость.
	\item увеличивается издержки производства на разработку таких процессоров -- их конечная стоимость возрастает.
\end{itemize}

Таким образом, в некоторых случаях, появляется необходимость в программной реализации ДСИ. Стоит отметить, что в конечном счёте, программная реализация всё равно использует другие аппаратные механизмы предоставляемые процессором \cite{aaa}.
 
\section{Существующие реализации ДСИ}

\subsection{Intel SGX}

Intel

\subsection{Keystone}

RISC V

\subsection{ARM TrustZone}

ARM TrustZone -- технология аппаратного обеспечения ДСИ, разрабатываемая компанией ARM. Большинство процессоров разработанных ARM имеют поддержку TrustZone \cite{comparsion-arm-intel}. Данная технология основана на разделении режимов работы процессора на два "мира": обычный мир (Normal World) и безопасный мир (Secure World). Процессор переключается в безопасный мир по запросу (с помощью специальной инструкции), при работе с конфиденциальными данными. Всё остальное время, процессор работает в режиме обычного мира. Процессоры с поддержкой данной технологии имеют способность разделять память, независимо от её типа, на ту, которая доступна только в безопасном мире, и ту, которую можно использовать в обычном мире. ARM предоставляют открытый исход программного обеспечения для поддержки данной аппаратной технологии -- ARM Trusted Firmware \cite{arm-tfa}.

\subsubsection{Режимы работы процессора}

\subsubsection{Обычный и безопасный мир}

Ключевой особенной ARM TrustZone является способность процессора переключаться между обычным и безопасным миром. Каждый из этих миров управляется собственной операционной системой, которые обеспечивают необходимую функциональность. Основное различие между этими ОС заключается в предоставляемых гарантия безопасности. В один момент времени, процессор может находиться только в одном из двух миров, что определяется значением специального бита NS (Non-Secure), бит является частью регистра Secure Configuration Register (SCR). Этот регистр доступен для периферии только для чтения, изменять его значение может лишь сам процессор. Когда процессор находится в обычном режиме исполнения кода, значение бита NS равно 1, и наоборот, когда процессор находится в безопасном мире, значение бита NS равно 0.

За связь между обычным и безопасным миром отвечает специальный механизм -- Secure Monitor. Он соединяет оба мира и является единственной точкой входа в безопасный мир. Для того, чтобы из обычного мира перейти в безопасный, существует специальная инструкция процессоров ARM -- Secure Monitor Call (SMC). При вызове данной инструкции процессор передает управление Secure Monitor. Тот в свою очередь готовим систему к переходу из одного мира в другой и передает управление соответствующей ОС. Инструкция SMC используется как для перехода из нормального мира в безопасный, так и для перехода из безопасного в нормальный. Некоторые прерывания или исключения могут быть настроены так, чтобы они так же проходили через Secure Monitor и были обработаны в безопасном мире. ARM предоставляет спецификацию Secure Monitor Call Calling Convention (SMCCC) \cite{smccc}, которая является стандартом при реализации вызовов SMC. 

На рисунке \ref{fig:trustzone-conceptual} представлена цонепутальная схема взаимодействия двух миров.

\begin{figure}[h]
	\centering
	\includegraphics[width=\textwidth]{img/arm-conceptual.pdf}
	\caption{Концептуальная схема взаимодействия двух миров для процессоров ARM Cotrex-A}
	\label{fig:trustzone-conceptual}
\end{figure}

Доверенная операционная система, так же как и обычная, может запускать приложения. Они, как и в обычном мире, обращаются к доверенной ОС при необходимости получения каких-либо ресурсов или обработки прерываний и исключений. Таким образом, Secure Monitor передаёт управление одному из доверенных приложений, и уже те, в свою очередь, обращаются к доверенной ОС.

ARM предоставляют открытый исходной код эталонной доверенной операционной системы, которая называется OP-TEE \cite{optee}. Global Platform предоставляет спецификацию для реализации API взаимодействия доверенных приложений \cite{teec-spec} с доверенной ОС \cite{tee-spec}.

Физически миры разделены таким образом, что часть регистров доступны только в безопасном мире. Периферия, например память, может быть настроена так, что она может быть доступна лишь в определенном мире. Технология TrustZone в нормальном режиме работы процессора не позволяет программному обеспечению получить доступ к аппаратным средствам, которые могут быть доступны лишь только в безопасном мире. 

При сборке компонентов системы, производитель устройства должен позаботиться о конфигурации периферии для работы с TrustZone:

\begin{itemize}
	\item если предполагается, что периферия может получать доступ к безопасному режиму исполнения, процессор и внешнее устройство должны быть соединены (помимо различных шин) линией NS. Получение сигнал NS=0 от процессора к периферии означает, что команда является доверенной (например операция чтения или записи).
	\item в обратном случае, линия NS может быть опущена. Предполагается, что такая периферия не имеет никаких привилегий, т.е. NS=1 всегда.
\end{itemize}

На рисунке \ref{fig:ns-bit} представлена схема взаимодействия процессора и периферии для поддержки TrustZone. Периферийные устройства №1 и №3 соединены линией NS, а устройство №2 нет.

\begin{figure}[h]
	\centering
	\includegraphics[width=\textwidth]{img/arm-ns.pdf}
	\caption{Пример взаимодействия процессора и периферии для поддержки TrustZone}
	\label{fig:ns-bit}
\end{figure}

Стоит отметить, что чаще всего не вся периферия соединена сигналом NS с процессором. Например, тот факт, что производитель устройства не соединил сигналом NS камеру и ядра процессора, полностью исключает возможность предоставления пользователю доступа к устройству с помощью технологии распознавания лица.

\subsubsection{Разделение памяти}

\subsubsection{Другие компоненты}

\section{Сравнение реализаций ДСИ}

\subsection{Критерии сравнения}

Для сравнения раннее описанных реализаций ДСИ были выделены следующие критерии:

\begin{itemize}
	\item К1 - безопасность;
	\item K2 - производительность;
	\item К3 - надежность механизма аттестации ДСИ;
	\item К4 - наличие открытого исходного кода.
\end{itemize}

\pagebreak
