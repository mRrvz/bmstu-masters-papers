\documentclass[12pt, a4paper]{article}

\include{preamble}

\begin{document}

\include{preambule}
	
\section{Отчет}
	
\subsection{Постановка задачи}
	
Разработать процедуру, вычисляющую определитель вещественной матрицы А порядка N (первый шаг -- приведение матрицы к верхнему треугольному виду). Обсновать проектное решение (выбор алгоритма). Обеспечить равномерную загрузку процессоров. Результат вывести в текстовый файл построчно. Использовать зависимость времени счета от размерности задачи и количества процессоров.
	
\subsection{Процедура, вычисляющая определитель матрицы}
	
В листинге \ref{lst:noParallel} представлена программа на ЯП С, вычисляющую определитель вещественной матрицы.
	
\begin{lstlisting}[label=lst:noParallel,caption=Вычисление определителя вещественной матрицы]
#include <stdio.h>
#include <stdlib.h>
#include <mpi.h>
#include <math.h>

#define N 512

static double **allocate_matrix(int size)
{
	int i, j;
	double **matrix;
	
	matrix = malloc((size) * sizeof(double *));
	for (i = 0; i < size; i++)
	matrix[i] = malloc((size) * sizeof(double));
	
	return matrix;
}

static void **fill_random_matrix(double **matrix, int size)
{
	int i, j;
	
	for (i = 0; i < size; i++)
	for (j = 0; j < size; j++)
	matrix[i][j] = (rand() % 10) + 1;
}

int main(int argc, char** argv)
{
	double **A = allocate_matrix(N);
	
	fill_random_matrix(A, N);
	
	double det = 1;
	double norm = 0;
	int i, j, k;
	
	for (k = 0; k < N - 1; k++) {

		double max_val = abs(A[k][k]);
		int max_row = k;
		for (i = k + 1; i < N; i++) {
			if (abs(A[i][k]) > max_val) {
				max_val = abs(A[i][k]);
				max_row = i;
			}
		}
		
		if (max_row != k) {
			for (j = k; j < N; j++) {
				double temp = A[k][j];
				A[k][j] = A[max_row][j];
				A[max_row][j] = temp;
			}
			det *= -1.0;
		}
		
		for (i = k + 1; i < N; i++) {
			double factor = A[i][k] / A[k][k];
			for (j = k; j < N; j++) {
				A[i][j] -= factor * A[k][j];
			}
		}
	}
	
	for (i = 0; i < N; i++) {
		det *= A[i][i];
	}
	
	printf("Determinant: %lf\n", det);
	
	return 0;
}

\end{lstlisting}
	
\subsection{Процедура, вычисляющая определитель матрицы параллельно}
	
В листинге \ref{lst:parallel} представлена программа на ЯП С с использованием средств библиотеки MPI, реализующая вычисление определителя вещественной матрицы параллельно.
	
\begin{lstlisting}[label=lst:parallel,caption=Параллельное вычисление определителя матрицы]
#include <stdio.h>
#include <stdlib.h>
#include <string.h>
#include <mpi.h>

#define N 100

static void print_matrix(double *matrix, size_t len)
{
	printf("\n");
	for (int row_idx = 0; row_idx < len; ++row_idx) {
		for (int col_idx = 0; col_idx < len; ++col_idx) {
			printf("%lf ", matrix[row_idx * len + col_idx]);
		}
		printf("\n");
	}
}

static void print_array(double *arr, int size)
{
	for (int i = 0; i < size; i++) {
		printf("%lf ", arr[i]);
	}
	
	printf("\n");
}

static void init_matrix(double *matrix)
{
	for (size_t row_idx = 0; row_idx < N; ++row_idx)
	for (size_t col_idx = 0; col_idx < N; ++col_idx)
	matrix[row_idx * N + col_idx] = (rand() % 10) + 1;
}

#define SWAP(t, a, b) do { t c = a; a = b; b = c; } while (0)

static void swap(double *matrix, size_t count, size_t row1, size_t row2)
{
	for (size_t i = 0; i < count; i++)
	SWAP(double, matrix[i * N + row1], matrix[i * N + row2]);
}

static double gaussian(double *matrix, double *send_buffer, int num_cols, int rank, int size)
{
	double m_determinant = 0;
	int cur_control = 0;
	size_t swaps = 0;
	size_t cur_row = 0;
	size_t cur_index = 0;
	size_t row_swap;
	double det_val = 1;
	
	for (size_t i = 0; i < N; i++) {
		if (cur_control == rank) {
			row_swap = cur_row;
			double max = matrix[cur_index * N + cur_row];
			
			for (size_t j = cur_row + 1; j < N; j++) {
				if (matrix[cur_index * N + j] > max) {
					row_swap = j;
					max = matrix[cur_index * N + j];
				}
			}
		}
		
		MPI_Bcast(&row_swap, sizeof(size_t), MPI_BYTE, cur_control, MPI_COMM_WORLD);
		if (row_swap != cur_row) {
			swap(matrix, num_cols, cur_row, row_swap);
			swaps++;
		}
		
		if (cur_control == rank)
		for (size_t j = cur_row; j < N; j++)
		send_buffer[j] = matrix[cur_index * N + j] / matrix[cur_index * N  + cur_row];
		
		MPI_Bcast(send_buffer, N, MPI_DOUBLE, cur_control, MPI_COMM_WORLD);
		for (size_t j = 0; j < N; j++)
		for (size_t k = cur_row + 1; k < N; k++)
		matrix[j * N + k] -= matrix[j * N + cur_row] * send_buffer[k];
		
		if (cur_control == rank) {
			det_val = det_val * matrix[cur_index * N + cur_row];
			cur_index++;
		}
		
		cur_control++;
		if (cur_control == size)
		cur_control = 0;
		
		cur_row++;
	}
	
	MPI_Reduce(&det_val, &m_determinant, 1, MPI_DOUBLE, MPI_PROD, 0, MPI_COMM_WORLD);
	
	if (swaps % 2)
	m_determinant = -m_determinant;
	
	return m_determinant;
}

static void sort_by_process(double *list2, double *list1, size_t size)
{
	size_t index = 0;
	
	for (size_t i = 0; i < size; i++) {
		for (size_t j = i; j < N; j += size) {
			list1[index] = list2[j];
			index++;
		}
	}
}

int main(int argc, char *argv[])
{
	int rank, size;
	size_t num_rows, num_cols;
	double start_time, end_time;
	double det;
	double *matrix, *matrix_cpy, *ptr;
	double *send_buffer, *recv_buffer;
	
	MPI_Init(&argc, &argv);
	MPI_Comm_rank(MPI_COMM_WORLD, &rank);
	MPI_Comm_size(MPI_COMM_WORLD, &size);
	
	if (!rank)
	num_rows = N;
	
	MPI_Bcast(&num_rows, sizeof(size_t), MPI_BYTE, 0, MPI_COMM_WORLD);
	
	num_cols = num_rows / size;
	
	matrix = malloc(sizeof(double) * N * N);
	matrix_cpy = malloc(sizeof(double) * N * N);
	ptr = matrix_cpy;
	
	if (!rank) {
		init_matrix(matrix);
		memcpy(matrix_cpy, matrix, sizeof(double) * N * N);
	}
	
	send_buffer = malloc(sizeof(double) * N);
	recv_buffer = malloc(sizeof(double) * num_cols);
	
	for (size_t i = 0; i < N; i++) {
		if (!rank) 
		sort_by_process(matrix_cpy, send_buffer, size);
		
		MPI_Scatter(send_buffer, num_cols, MPI_DOUBLE, recv_buffer, num_cols, MPI_DOUBLE, 0, MPI_COMM_WORLD);
		
		if (!rank)
		matrix_cpy += N;
		
		for (size_t j = 0; j < num_cols; j++)
		matrix[j * N + i] = recv_buffer[j];
	}
	
	free(recv_buffer);
	
	MPI_Barrier(MPI_COMM_WORLD);
	start_time = MPI_Wtime();
	
	det = gaussian(matrix, send_buffer, num_cols, rank, size);
	
	MPI_Barrier(MPI_COMM_WORLD);
	end_time = MPI_Wtime();
	
	if (!rank) {
		printf("Determinant: %lf\n", det);
		printf("MPI Time: %lf\n", end_time - start_time);
	}
	
	free(matrix);
	free(ptr);
	free(send_buffer);
	
	MPI_Finalize();
	return 0;
}

\end{lstlisting}
	
\subsection{Оценка эффективности параллельной реализации алгоритма}
В целях оценки эффективности параллельной реализации алгоритма будут рассмотрены показатели ускорения, получаемого при использовании параллельного алгоритма для некоторого количества процессоров по сравнению с последовательным вариантом выполнения на различных размера матрицы.
	
Время исполнения было замерено с использованием внутренних средств библиотеки MPI.
	
Оценка эффективности и ускорения проводится для выполняемого на 1, 2, 4, 5, 8 и 10 процессорах при размерности квадратной матрицы 800 элементов и 1600.
	
\subsubsection{Оценка ускорения.}
	
На графике \ref{img:first} представлена зависимость значения ускорения от количества используемых процессоров при размерах матрицы 800.
	
\begin{figure}[H]
		\centering
		\includegraphics[width=\textwidth]{img/fst.png}
		\caption{ Зависимость значения ускорения от количества используемых процессоров (размер матрицы 800) }
		\label{img:first}
\end{figure}
	
Из представленного графика видно, что пик ускорения достигается при количестве процессоров, равному 8.

На графике \ref{img:second} представлена зависимость значения ускорения от количества используемых процессоров при размерах матрицы 1600.

\begin{figure}[H]
	\centering
	\includegraphics[width=\textwidth]{img/snd.png}
	\caption{ Зависимость значения ускорения от количества используемых процессоров (размер матрицы 1600)}
	\label{img:second}
\end{figure}

Из представленного графика видно, что пик ускорения достигается при количестве процессоров, равному 5.
	
\end{document}
