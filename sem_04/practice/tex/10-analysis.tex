\section{Оглавление ВКР}

\textbf{СОДЕРЖАНИЕ}

\textbf{ВВЕДЕНИЕ}

\textbf{1 Аналитический раздел}

1.1 Анализ предметной области

1.1.1 Кольца привилегий

1.1.2 Доверенная среда исполнения

1.2 Существующие реализации ДСИ

1.2.1 ARM TrustZone

1.2.2 Intel SGX 

1.2.3 Keystone

1.3 Виды угроз безопасности

1.3.1 Физические атаки 

1.3.2 Атаки на привилегированное ПО

1.3.3 Программные атаки на периферию

1.3.4 Трансляция адресов

1.4 Сравнение реализаций ДСИ

1.4.1 Критерии сравнения 

1.4.2 Сравнение безопасности 

1.4.3 Сравнение производительности

1.4.4 Итоговая таблица

1.5 Виртуализация в процессорных системах ARM

1.5.1 Виртуализация ARM TrustZone

1.6 Постановка задачи

\textbf{2 Конструкторская часть}

2.1 Разработка метода программной реализации доверенной среды исполнения

2.1.1 Модуль защищенного отображения памяти

2.1.2 Модуль блокировки потока управления

2.1.3 Модуль переключения контекста

2.1.4 Индивидуальное рабочее окружение

2.2 Формальное описание метода

2.2.1 Описание доверенной загрузки

2.2.2 Описание защиты и переключение контекста выполнения

2.2.3 Описание разделения аппаратных ресурсов

\textbf{3 Технологическая часть}

3.1 Выбор операционной системы

3.2 Выбор средств виртуализации

3.3 Сборка программного обеспечения

3.4 Требования к вычислительной системе

3.5 Структура программного обеспечения

3.5.1 Модификация гипервизора

3.5.2 Обработчик переключения контекста

3.5.3 Функция обработки элементов таблицы страниц

\textbf{4 Исследовательская часть}

4.1 Методика проведения исследования

4.2 Сравнение количества машинных инструкций с аппаратной реализацией

4.2.1 Сравнение при выполнение ключевых задач

4.2.2 Сравнение с использованием пользовательских приложений

4.2.3 Сравнение с использованием серверных приложений\newline

\textbf{ЗАКЛЮЧЕНИЕ}

\textbf{СПИСОК ИСПОЛЬЗОВАННЫХ ИСТОЧНИКОВ}

\pagebreak
