\documentclass[12pt, a4paper]{article}

\include{preamble}

\begin{document}

\include{preambule}


\begin{titlepage}
	\noindent\begin{minipage}{0.05\textwidth}
		\includegraphics[scale=0.3]{img/bmstu.png}
	\end{minipage}
	\hfill
	\begin{minipage}{0.85\textwidth}\raggedleft
		\begin{center}
			\fontsize{12pt}{0.3\baselineskip}\selectfont \textbf{Министерство науки и высшего образования Российской Федерации \\ Федеральное государственное бюджетное образовательное учреждение \\ высшего образования \\ <<Московский государственный технический университет \\ имени Н.Э. Баумана \\ (национальный исследовательский университет)>> \\ (МГТУ им. Н.Э. Баумана)}
		\end{center}
	\end{minipage}
	
	\begin{center}
		\fontsize{12pt}{0.1\baselineskip}\selectfont
		\noindent\makebox[\linewidth]{\rule{\textwidth}{4pt}} \makebox[\linewidth]{\rule{\textwidth}{1pt}}
	\end{center}
	
	\begin{flushleft}
		\fontsize{12pt}{0.8\baselineskip}\selectfont 
		
		ФАКУЛЬТЕТ \textbf{<<\textbf{Информатика и системы управления}>> \hfill}
		
		КАФЕДРА \textbf{\mbox{\hspace{4mm}} <<\textbf{Программное обеспечение ЭВМ и информационные технологии}>> \hfill}
	\end{flushleft}
	
	\vfill
	
	\begin{center}
		\fontsize{20pt}{\baselineskip}\selectfont
		
		\textbf{{Отчет по лабораторной работе}}
		
		\textbf{{по курсу}}
		
		\textbf{<<Программирование специализированных вычислительных устройств>>}
	\end{center}
	
		\fontsize{18pt}{0.6cm}\selectfont 
		""\newline\newline\newline\newline\newline\newline\newline
		\textbf{Студент:} Романов А. В.\newline
		\textbf{Преподаватель:} Ковтушенко А. П.\newline
		\textbf{Группа:} ИУ7-42М\newline
	
	\vfill
	
	\begin{table}[h!]
		\fontsize{12pt}{0.7\baselineskip}\selectfont

		
		\vspace{\baselineskip}
		
	\end{table}
	
	\vfill
	
	\begin{center}
		\normalsize \textit{\textbf{2024} г.}
	\end{center}
\end{titlepage}
	
\section{Постановка задачи}

В ходе выполнения лабораторной работы требуется проанализировать
зависимость времени работы программы, выполняющей умножение матриц при
помощи технологии CUDA в зависимости от размеров матриц, соотношения их
сторон и их расположения в памяти (4 случая).
Программа выполняет умножение матриц, при этом поддерживает 4
варианта расположения матриц в памяти. Варианты отличаются возможным
транспонированием одной или обоих матриц.

\section{Результаты}

В начале выполнялось тестирование для квадратных матриц. На
рисунке 1 приведён график зависимости ускорения вычислений умножения в
зависимости от размеров квадратных матриц с учётом различия расположения
матриц в памяти.
Далее различающиеся длины сторон матриц были зафиксированы на
значении 1024, и производилось изменение общей стороны матрицы. График
зависимости ускорения вычислений умножения в зависимости от длины общей
стороны с учётом различия расположения матриц в памяти приведён на
рисунке 2.

\begin{figure}[H]
	\centering
	\includegraphics[width=\textwidth]{img/1.pdf}
	\caption{Зависимость ускорения от размера квадратной матрицы}
	\label{img:1}
\end{figure}

\begin{figure}[H]
	\centering
	\includegraphics[width=\textwidth]{img/2.pdf}
	\caption{Зависимость ускорения от длинны общей стороны}
	\label{img:2}
\end{figure}
	
\end{document}
