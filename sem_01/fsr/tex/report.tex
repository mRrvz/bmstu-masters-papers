\documentclass[12pt, a4paper]{article}

\include{preamble}

\begin{document}


\noindent УДК~004.94

\hfill

\noindent \textbf{Объединение содержимого одинаковых страниц оперативной памяти сжатых модулем ядра Linux}

\noindent А.~В.~Романов$^{1}$\hfill romanov.alexey2000@gmail.com

\noindent $^{1}$МГТУ им.~Н.~Э.~Баумана, Москва, Россия

\hfill

\noindent \textbf{Аннотация}

\noindent Статья посвящена 

\noindent \textbf{Ключевые слова}

\noindent \textit{}

\hfill

\section*{Введение}

\section{Постановка задачи}

\section{Структуры данных и схема работы модуля ядра zram}

\section{Алгоритм объединения содержимого страниц оперативной памяти}

\section{Сравнительный анализ результатов работы разработанного алгоритма}

\section*{Заключение}


\begin{thebibliography}{5}
	%\bibitem{1} Градов В. М. Курс лекций по моделированию. МГТУ им. Н. Э. Баумана.
	%\bibitem{2} Абдурагимов Э. И. Метод сеток решения задачи Дирихле для уравнения Пуассона. ДГУ
\end{thebibliography}

\noindent \textbf{Романов Алексей Васильевич} — студент, МГТУ им. Н. Э. Баумана, кафедра «Программное обеспечение ЭВМ и информационные технологии».


\end{document}
