\section*{СПИСОК ИСПОЛЬЗОВАННЫХ ИСТОЧНИКОВ}
\addcontentsline{toc}{section}{СПИСОК ИСПОЛЬЗОВАННЫХ ИСТОЧНИКОВ}

\begingroup
\renewcommand{\section}[2]{}
\begin{thebibliography}{}
	\bibitem{zsmalloc}
	zsmalloc — The Linux Kernel documentation [Элекронный ресурс]. - Режим доступа:
	https://www.kernel.org/doc/html/v5.5/vm/zsmalloc.html
	
	\bibitem{kernel-development}
	Ядро Linux. Описание процесса разработки. Третье издание, 2019. Роберт Лав. с. 25 - 36.
	
	\bibitem{bootlin}
	Linux Kernel Source Code — Elixir Bootlin [Элекронный ресурс]. - Режим доступа:
	https://elixir.bootlin.com/linux/latest/source/mm/zsmalloc.c
	
	\bibitem{hash-table}
	hash table — IBM [Элекронный ресурс]. - Режим доступа:
	https://www.ibm.com/docs/en/cics-ts/5.4?topic=overview-hash-table
	
	\bibitem{rbtree}
	Red-Black Tree — Microsoft [Элекронный ресурс]. - Режим доступа:
	https://learn.microsoft.com/en-us/openspecs/windows\_protocols/ms-cfb/d30e462c-5f8a-435b-9c4c-cc0b9ea89956
	
	\bibitem{spinlock}
	What is a spin lock?  — IBM [Элекронный ресурс]. - Режим доступа:
	https://www.ibm.com/support/pages/what-spin-lock
	
	\bibitem{slab-cache}
	 Slab Allocator —  The Linux Kernel documentation [Элекронный ресурс]. - Режим доступа:
	https://www.kernel.org/doc/gorman/html/understand/understand011
\end{thebibliography}
\endgroup

\pagebreak