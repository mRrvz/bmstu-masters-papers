\section*{ВВЕДЕНИЕ}
\addcontentsline{toc}{section}{ВВЕДЕНИЕ}

Существует несколько способов увеличения количества оперативной памяти. Один из способов заключается в физическом увеличении количества планок ОЗУ в системе. Данный способ подразумевает покупку и установку планок ОЗУ, что требует денежных затрат. Кроме физического способа увеличения количества памяти, существуют программные способы увеличения количества ОЗУ, например, сжатие данных. Данный способ требует только вычислительные мощности CPU. Кроме того, к программным способам, можно отнести дедупликацию данных -- объединение участков в памяти, содержащих одинаковые данные, в одно целое. Два последних способа можно объединить и получить ещё один наиболее эффективный способ увеличения количества оперативной памяти: дедупликация сжатых данных. Целью данной научно-\\исследовательской работы является разработка метода объединения одинаковых объектов для распределителя памяти zsmalloc в ядре Linux:

Для достижения поставленной цели необходимо выполнить следующие задачи:

\begin{itemize}
	\item изложить особенности предложенного метода;
	\item сформулировать и описать основные этапы метода в виде схем алгоритмов;
	\item описать структуры данных, используемые в разработанном алгоритме.
\end{itemize}

\pagebreak
